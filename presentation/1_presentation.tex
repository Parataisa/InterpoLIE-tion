\documentclass[11pt,t,usepdftitle=false,aspectratio=169]{beamer}

\usetheme[nototalframenumber,foot,logo]{uibk}

%% information for the title page
\title[Erkennung von Resampling]{Erkennung von Resampling}
\subtitle{InterpoLIE-tion - Catching lies through interpolation analysis}

\author[Dominik Barbist, Lukas Egger]{Dominik Barbist, Lukas Egger \\ \vspace{0.5em}}
\footertext{Digitale Forensik mit Bild- und Videodaten}
\date{\today}

\headerimage{3}

\begin{document}

%% this sets the first PDF bookmark and triggers generation of the title page
\section{Introduction}

%% Custom overview slide since TOC doesn't show subsections
\begin{frame}
	\frametitle{Übersicht}
	\begin{enumerate}
		\item \textbf{Einführung}
		\begin{itemize}
			\item Motivation
		\end{itemize}
		\item \textbf{Problemstellung}
		\begin{itemize}
			\item Resampling Detection
		\end{itemize}
		\item \textbf{Lösungsansätze}
		\begin{itemize}
			\item Exposing Digital Forgeries by Detecting Traces of Resampling
			\item Fast and Reliable Resampling Detection by Spectral Analysis
			\item Blind Authentication Using Periodic Properties of Interpolation
			\item Detection of Linear and Cubic Interpolation in JPEG Compressed Images
			\item Normalized Energy Density-Based Forensic Detection
			\item An SVD Approach to Forensic Image Resampling Detection
		\end{itemize}
	\end{enumerate}
\end{frame}

\subsection{Einführung}

\begin{frame}
	\frametitle{Einführung}
	\begin{itemize}
		\item \textbf{Motivation:} 
		\begin{itemize}
			\item Digitale Forensik
			\item Manipulation von Bild- und Videodaten
			\item Resampling als häufige Manipulationstechnik
			\item Notwendigkeit der Erkennung von Resampling
		\end{itemize}
	\end{itemize}
\end{frame}

\subsection{Problemstellung}

\begin{frame}
	\frametitle{Problemstellung}
	\begin{itemize}
		\item \textbf{Resampling Detection:} 
		\begin{itemize}
			\item Resampling als häufige Manipulationstechnik
			\item Herausforderungen bei der Erkennung
			\item Notwendigkeit robuster und effizienter Methoden
		\end{itemize}
	\end{itemize}
\end{frame}

\subsection{Lösungsansätze}

\begin{frame}
	\frametitle{Lösungsansätze}
	\begin{itemize}
		\item \textbf{Exposing Digital Forgeries by Detecting Traces of Resampling:} 
		\begin{itemize}
			\item Identifikation von Resampling-Spuren
			\item Analyse der Auswirkungen auf Bilddaten
		\end{itemize}
		\item \textbf{Fast and Reliable Resampling Detection by Spectral Analysis:} 
		\begin{itemize}
			\item Nutzung der Spektralanalyse zur Erkennung von Resampling
			\item Effizienz und Zuverlässigkeit der Methode
		\end{itemize}
		\item \textbf{Blind Authentication Using Periodic Properties of Interpolation:} 
		\begin{itemize}
			\item Authentifizierung ohne Vorwissen über das Bild
			\item Periodische Eigenschaften der Interpolation nutzen
		\end{itemize}
	\end{itemize}
\end{frame}

\begin{frame}
	\frametitle{Lösungsansätze (Fortsetzung)}
	\begin{itemize}
		\item \textbf{Detection of Linear and Cubic Interpolation in JPEG Compressed Images:} 
		\begin{itemize}
			\item Spezielle Fokussierung auf JPEG-Bilder
			\item Unterscheidung zwischen linearer und kubischer Interpolation
		\end{itemize}
		\item \textbf{Normalized Energy Density-Based Forensic Detection:} 
		\begin{itemize}
			\item Energie-Dichte-Analyse zur Forensik
			\item Normalisierung für verbesserte Genauigkeit
		\end{itemize}
		\item \textbf{An SVD Approach to Forensic Image Resampling Detection:} 
		\begin{itemize}
			\item Singular Value Decomposition (SVD) zur Resampling-Erkennung
			\item Mathematische Grundlagen und Implementierung
		\end{itemize}
	\end{itemize}
\end{frame}

\subsection{Zusammenfassung}

\begin{frame}
	\frametitle{Zusammenfassung}
	\begin{itemize}
		\item Resampling Detection ist ein wichtiger Aspekt der digitalen Forensik
		\item Verschiedene Ansätze bieten robuste und effiziente Lösungen
		\item Zukünftige Entwicklungen könnten die Genauigkeit und Anwendbarkeit weiter verbessern
	\end{itemize}
\end{frame}

%% to show a last slide similar to the title slide
\title[Erkennung von Resampling]{Erkennung von Resampling}
\subtitle{InterpoLIE-tion - Catching lies through interpolation analysis}
\section{Vielen für Ihre Aufmerksamkeit!}


\end{document}